\documentclass[a4paper,10pt]{article}
\usepackage{graphicx}
\usepackage{hyperref}
\usepackage{listings}

\title{Security in Software Applications - Project 1}
\author{LastName Matricola}
\date{A.A. 2024/2025}

\begin{document}
\maketitle

\section{Flawfinder: Strengths and Weaknesses}
Flawfinder is a useful static analysis tool for detecting potential security vulnerabilities in C/C++ code. Its main strengths include:
\begin{itemize}
    \item Quick analysis with a ranked severity system.
    \item Identifies well-known insecure functions.
    \item Provides CWE references for better understanding.
\end{itemize}
However, the tool also has limitations:
\begin{itemize}
    \item High false positive rate, requiring manual inspection.
    \item Does not detect logical vulnerabilities or runtime issues.
    \item Some warnings are too generic, making prioritization difficult.
\end{itemize}

\section{Analysis of Warnings}
Below are the key warnings found and their analysis:
\begin{itemize}
    \item \textbf{strcpy (line 42) [4]}: Unsafe due to lack of bounds checking. Should be replaced with \texttt{snprintf} or \texttt{strncpy} with explicit null termination.
    \item \textbf{fprintf (line 56) [4]}: Possible format string vulnerability. The format should be constant to prevent attacks.
    \item \textbf{Statically-sized arrays (lines 8, 28, 33) [2]}: Risk of buffer overflow. Dynamic allocation or bound checking is recommended.
    \item \textbf{strcat (line 35) [2]}: Does not check destination size, risking overflow. \texttt{strncat} with proper length should be used.
    \item \textbf{strncpy (lines 9, 34) [1]}: May not null-terminate. Ensure explicit \texttt{dst[size-1] = '\\0'}.
    \item \textbf{read (lines 17, 22) [1]}: Buffer boundaries should be checked before use.
\end{itemize}

\section{Additional Vulnerabilities}
Flawfinder did not flag:
\begin{itemize}
    \item \textbf{Incorrect use of strlen in func1}: The buffer size calculation is incorrect, potentially leading to overflows.
    \item \textbf{Unhandled exceptions in func3}: The try-catch block is invalid in C and will not compile.
    \item \textbf{Infinite loop risk in main}: The loop at the end of \texttt{main} accesses an array out of bounds.
\end{itemize}

\section{Corrected Code}
The following modifications were made to secure the code:
\begin{lstlisting}[language=C]
#include <stdlib.h>
#include <string.h>
#include <stdio.h>
#include <ctype.h>

void func1(const char *src) {
    size_t len = strlen(src);
    char *dst = malloc(len + 1);
    if (dst) {
        strncpy(dst, src, len);
        dst[len] = '\0';
        free(dst);
    }
}

void func2(int fd) {
    char *buf;
    size_t len;
    if (read(fd, &len, sizeof(len)) != sizeof(len) || len > 1024)
        return;
    buf = malloc(len + 1);
    if (buf) {
        if (read(fd, buf, len) == len) {
            buf[len] = '\0';
        }
        free(buf);
    }
}

void func3() {
    char buffer[1024];
    printf("Please enter your user id:");
    if (fgets(buffer, sizeof(buffer), stdin)) {
        if (!isalpha(buffer[0])) {
            char errormsg[1044];
            snprintf(errormsg, sizeof(errormsg), "%s is not a valid ID", buffer);
        }
    }
}

void func4(const char *foo) {
    char *buffer = (char *)malloc(11);
    if (buffer) {
        strncpy(buffer, foo, 10);
        buffer[10] = '\0';
        free(buffer);
    }
}

int main() {
    func4("safe_input");
    func3();
    return 0;
}
\end{lstlisting}

\section{Verification}
After running Flawfinder on the corrected code, no critical vulnerabilities were reported.

\end{document}
